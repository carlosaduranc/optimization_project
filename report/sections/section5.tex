\section{Conclusions}
\label{sec:conclusions}

During the length of this work, three different optimization applications where developed and applied to a study of a thermal network of an apartment located in Leuven, Belgium. \\

First, a parameter estimation problem was developed with the goal of identifying the parameters of a model based off of recorded data. This was done with a `reverse engineering' approach since a model structure was used to generate the data and was then used to fit said data by means of a non-linear least squares problem. This identification resulted in very accurate results for all three parameters, allowing for the identified model to be used in the consecutive steps.\\

Subsequently, an optimal control problem was developed with the goal of optimizing the heating deployment within the thermal zone while maintaining the required thermal comfort levels. This was done by means of a primal problem statement which resulted in an improved but rigid solution. This indicates that the system, while achieving the desired results, cannot be certain to not be unphysical since, for any instance on which the inequality constraints are not met, the program crashes.\\

Finally, a relaxed approach to the optimal control problem was developed with the goal of tackling the shortcomings of the primal solution. For this, a slack variable was introduced alongside a penalty factor which allowed for the system to become more flexible. Namely, the system became more physical since it allows for external factors to affect the zone temperature while still heavily penalizing it and adjusting the controls in a way that the constraints are met as soon as possible. Additionally, this solution further decreased the final energy consumption for the apartment by approximately half of the baseline case. 